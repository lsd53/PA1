\section{Introduction}\label{introduction}

This document serves as a detailed explanation of our MIPS processor. It
explains the core logic behind each stage at a low level, preceded by a
high-level view of the design.

The design document is intended as an aid for the simulated Logisim
processor. It clarifies the design of the circuit so that each stage of
the processor can be understood in full. The document assumes a basic
knowledge of logic gates and pieces of a processor, e.g.~a register.

The document begins at the top level, describing the processor as a set
of stages. Each stage will then be explained in detail, and each
subcircuit, such as the program counter incrementer, will be covered.

Note that for any \(N\)-bit binary number, bit 0 refers to the least
significant bit and bit \(N-1\) refers to the most significant bit. For
example, bit 3 is the 4th least significant bit.

\section{Overview}\label{overview}

The MIPS processor is designed as a 5-stage pipelined processor. Here is
a basic overview of the stages;

\begin{enumerate}
\def\labelenumi{\arabic{enumi}.}
\itemsep1pt\parskip0pt\parsep0pt
\item
  \emph{Fetch} The instruction is fetched from the program memory, and
  the program counter is advanced.
\item
  \emph{Decode} The instruction from the previous stage is separated
  into pieces. Any values stored in the Register File are fetched and
  passed on.
\item
  \emph{Execute} The pieces from the Decode stage are plexed to select
  the proper values, which are subsequently passed through the ALU or
  similar subcircuit. The result of this operation is passed to the next
  stage.
\item
  \emph{Memory} This stage of the processor was unimplemented for this
  project.
\item
  \emph{Writeback} The values computed in the Execute stage are written
  back to the Register File.
\end{enumerate}

Separating each stage is a set of registers (`pipline registers'). These
hold values between stages, such as the writeback register.

The stages are controlled by the clock. Most stages move forward on the
rising edge of the clock; the Register File writeback is conducted on
the falling edge. Note that values such as the writeback register are
passed from the Decode stage to the Writeback stage without
interference. Additionally, any control flow instructions, such as jumps
and branches, are ignored. This is simulated by disabling the writeback
stage (no register is overwritten).

\section{Fetch Stage}\label{fetch-stage}

This stage controls which instruction in the program to execute. It
fetches instructions from the program ROM, or read-only memory. In order
to keep track of execution of the program, there is a 32-bit register
called the Program Counter (PC). Because there were no jumps or branches
for this project, the PC was controlled solely by the Incrementer.

!(./images/fetch.png){[}The Fetch stage{]}

\subsection{PC Incrementer}\label{pc-incrementer}

After executing an instruction, the program counter needs to be
incremented by 4. This is done using a one-bit adder. In order to
prevent using the same adder 4 times in series, the following subcircuit
was created:

!(./images/pcplus.png){[}The Program Counter incrementer{]}

By removing the first two bits of the program counter before adding 1,
the program counter is automatically incremented by 4 with just one
operation. The two bits that were initially removed are then appended
back on.

The program ROM returns a 32-bit instruction, which is written to the
IF/ID register.

\section{Decode Stage}\label{decode-stage}

The decode stage serves two purposes. It retrieves values from the
Register File and partly decodes the instruction itself.

image: decode stage

\subsection{Retrieving Register
Values}\label{retrieving-register-values}

The portion of the instruction carrying the register number for A and B
remains consistent for every register operation. The register number for
A is always represented by bits 21-25 inclusive, and the register number
for B is represented by bits 16-20, inclusive. Note that if the
instruction requested an operation with an immediate, bits 16-20 would
instead represent which register to write back to. This discrepancy is
covered in the next section, Decoding the Instruction.

\subsection{Decoding the Instruction}\label{decoding-the-instruction}

Although the Fetch stage retrieves values from the Register File, it
also partly decodes the instruction. It does so by separating the 32-bit
instruction into smaller sets of bits. Then, it passes each of these
sets to the ID/EX pipeline registers, plexing between several values if
necessary.

There are 7 pipeline registers between the Decode and Execute stage. Two
of these registers represent the operands for the operation, A and B
(see Retrieving Register Values above).

The next register contains the immediate value. In immediate operations,
this value is represented by bits 0-15, inclusive, so these bits are
sent directly to this register.

The fourth register down contains the writeback register. If the
operation is an immediate operation, the writeback register is bits
16-20, inclusive, otherwise the writeback register is bits 11-15. By
looking carefully at the opcodes for the immediate and register
operations, it becomes apparent that immediate operations always have a
1 in bit 29 of the instruction, and register operations always have a 0.
Therefore, this bit is sent to the multiplexer as the selector.

image: Rw register

Register 5 holds the opcode for the current operation; this is simply
bits 31-26, inclusive. These bits can be passed directly to the
register.

The next register holds the shift amount for left and right shifts if it
is represented by an immediate. If it is, bits 6-10 represent this
value, so these are passed directly to this register.

Lastly, the ALU function is passed as well; this is represented by bits
0-5.

At this point, there are many redundant values - the register value for
B could be sent to both pipeline register B as well as the writeback
register, whereas only one will be used. The logic to decide between
which is used lies in the Execute stage. Additionally, because each of
these fetch/decode operations are conducted in parallel, the performance
of the processor is not relinquished.

\section{Execute Stage}\label{execute-stage}

The Execute stage is where instructions are evaluated and computed. This
stage contains the ALU, or arithmetic logic unit, which computes a
variety of operations including logical shifts and sums.

During this stage, the pipeline registers between the Decode and Execute
stage are multiplexed so that the correct inputs are sent to the ALU.
The selectors for these multiplexers were chosen by identifying patterns
in the opcodes and \texttt{func} register, then sending specific bits
from these numbers. These patterns are identified in detail below.

\subsection{Choosing an Operation
Subcircuit}\label{choosing-an-operation-subcircuit}

Not every instruction from the ROM uses the ALU. For this reason, there
are several computations that occur in parallel with the ALU; the proper
output from these subcircuits is then chosen. All of the operations can
be divided into the following subsections:

\begin{longtable}[c]{@{}ll@{}}
\toprule
Operation & Subcircuit\tabularnewline
\midrule
\endhead
ANDI & ALU\tabularnewline
ADDIU & ALU\tabularnewline
ORI & ALU\tabularnewline
XORI & ALU\tabularnewline
ADDU & ALU\tabularnewline
SUBU & ALU\tabularnewline
AND & ALU\tabularnewline
OR & ALU\tabularnewline
XOR & ALU\tabularnewline
NOR & ALU\tabularnewline
SLL & ALU\tabularnewline
SRL & ALU\tabularnewline
SRA & ALU\tabularnewline
SLLV & ALU\tabularnewline
SRLV & ALU\tabularnewline
SRAV & ALU\tabularnewline
SLT & Signed comparator\tabularnewline
SLTI & Signed comparator\tabularnewline
STLU & Unsigned comparator\tabularnewline
STLIU & Unsigned comparator\tabularnewline
MOVZ & Equality\tabularnewline
MOVN & Equality\tabularnewline
\bottomrule
\end{longtable}

As can be seen, the majority of the operations do utilize the ALU.

\subsubsection{How to decide which uses
ALU}\label{how-to-decide-which-uses-alu}

\section{Memory Stage}\label{memory-stage}

This stage remained unimplemented in this project. However, pipeline
registers were still inserted to simulate this fourth stage in the
processor.

\section{Writeback Stage}\label{writeback-stage}

The writeback stage is possibly the fastest stage in the pipeline.
During the writeback stage, the value held by the 32-bit MEM/WB pipeline
register B is written to the register denoted by pipeline register Rw.

The register writeback also occurs during the falling edge of the clock
cycle. For more information on why, refer to Resolving Data Hazards
below.

\section{Resolving Data Hazards}\label{resolving-data-hazards}

Data hazards occur when an instruction attempts to use a register whose
value has not yet been updated. For example:

\begin{Shaded}
\begin{Highlighting}[]
\KeywordTok{ADDI} \DataTypeTok{$1}\NormalTok{, }\DataTypeTok{$0}\NormalTok{, }\DecValTok{100}
\KeywordTok{ADDI} \DataTypeTok{$2}\NormalTok{, }\DataTypeTok{$1}\NormalTok{, }\DecValTok{1}
\end{Highlighting}
\end{Shaded}

By the time the second instruction is executed, the processor will not
have written the new value to register \texttt{\$1}. Therefore, in order
to account for such hazards, the proper value from the register must be
forwarded to the appropriate stage.

!(./images/forwarding.png){[}Forwarding logic{]}

With this implementation of forwarding, the values from the following
stages are sent back to the Execute stage. The proper value is chosen
via a multiplexer. The deciding input for the multiplexer is sent from
the forwarding unit:

image: forwarding unit

This unit takes in 3 inputs, each a register number. The registers are
compared; if the register being read by the current instruction matches
the register being written by a previous instruction, there is a data
hazard. The forwarding unit detects this and outputs a 1 for that bit, a
0 for the other output. This bit can then be sent directly to a
multiplexer to select the proper output.

\section{Summary}\label{summary}

Overall, the MIPS processor conducts a wide range of basic instructions.
It can handle data hazards through forwarding logic and conducts many
operations simultaneously due to its 5-stage pipelined design.
